%\documentclass{article}
\documentclass[nofootinbib,preprint,aps]{revtex4-1}
\usepackage{hyperref}
\usepackage{amsmath}
\usepackage{csquotes}
\usepackage{gensymb}
\usepackage{graphicx}
\graphicspath{ {images/} }
\usepackage{physics}
\usepackage{color}
\newcommand{\red}[1]{\textcolor{red}{\bf #1}}
\begin{document}
\title{Nuclear Waste Management and Disposal}

\author{Yuping Huang}%
\affiliation{Department of Physics and Astronomy, Carleton College}

\date{\today}%
\begin{abstract}
    As nuclear reactors around the world keep operating with no disposal method available for highly
    radioactive nuclear waste, how we manage nuclear waste is becoming more important. In this work,
    we examine a few scientific and technical related to nuclear waste management and disposal.
    The principles of radioactivity and nuclear fission helps us understand the source and properties
    of different types of nuclear waste. Focusing on the highly radioactive spent fuel, the rest
    of this work outlines the advantages and drawbacks of different spent fuel management and disposal
    methods. We study the methods used for performance assessments for Yucca Mountain Nuclear Waste Repository,
    especially in the context of volcanic disruption. Finally, we discuss some possible options for 
    nuclear waste management and disposal after
    the termination of the Yucca Mountain project.\\
    Word count: 6592\\
\end{abstract}
\maketitle
\tableofcontents
\newpage

\section{Introduction}
Nuclear power continues to play an important role in global energy supply.
The International Atomic Energy Agency (IAEA) estimates that nuclear power
consumption will increase by $40\%$ by 2030.\cite{iaea12}
Whereas the ``front end'' of the nuclear fuel cycle, such as reactor physics and power output, draws
the most attention from the public, the ``back end'' of the fuel cycle, i.e. waste management and
disposal, deserves equal, if not more attention.
Currently, the IAEA estimates that there are over 10,000 metric tons of highly radioactive nuclear waste 
generated worldwide annually,
most of which is spent nuclear fuel from nuclear reactors.\cite{iaea08, r12}
The world will have over $457,000$ metric tons of highly radioactive waste stock by the end of 2020.\cite{r12}
Highly radioactive waste remains hazardous for up to $10^6$ years after it is produced and thus
requires extremely careful disposal methods.
Nevertheless, there is currently no permanent repository in the world to dispose of these highly radioactive
waste.\footnote{The Waste Isolation Pilot Plant (WIPP) in New Mexico is designed to be a permanent radioactive
    waste repository, but only
for waste classified as Transuranic, which, by the Nuclear Regulatory Commission standard, does not have particularly
high radioactivity but only long half-life (See Sec.~\ref{sec:tru}). In particular, WIPP does not accept spent nuclear
fuel.}  

It is alarming that more than $75\%$ of the spent nuclear fuel worldwide is stored on-site at the nuclear power plants
in water pools that have exceeded their designed capacity and are reaching their limiting capacity.\cite{aa12}
Two weeks after the devastating earthquake and tsunami hit the Fukushima Daiichi Nuclear Power Plant in Japan,
the water pool cooling system was cut off from power. Officials feared that
the highly radioactive waste in the pools would catch fire and send radioactive smoke
across a much larger part of Japan, including Tokyo.\cite{s16} For some ``fortuitous'' reasons, the fear did not come true.
However, the Nuclear Regulatory Commission (NRC), the independent government agency
of the U.S. government overseeing nuclear energy safety,
estimates that a major fire in a U.S. spent fuel pool is expected to
displace 3.4 million people in an area larger than New Jersey.\cite{s16}

Evidently, nuclear waste management and disposal is becoming a increasingly urgent matter globally.
In light of the high stake,
the uncertainty of the stability of a geological repository over a extremely long timescale has evolved into 
a topic not only of scientific contention but also of policy debate.
How well is the predicative power of science? How much trust should we put into the uncertainty estimates
(i.e. \red{how well can science predict its predicative power})?

In this work, we will attempt to examine some of the scientific issues underlying nuclear waste management
and disposal. We will start by building up some
language and knowledge necessary to understand how nuclear waste is generated (Section~\ref{sec:phys}),
then apply those principles to study
the physical properties of nuclear waste (Section~\ref{sec:waste}), its management and temporary storage
(Section~\ref{sec:temp}), and the debates around its eventual disposal (Section~\ref{sec:disposal}).

\section{Radioactivity and nuclear fission}
\label{sec:phys}
    Here we detail the nuclear physics underlying most of the discussions
    in this work. We will apply these results in our discussion of nuclear wastes.
    \subsection{Radioactivity}
    Something about radioactivity. The derivations and in this section follows chapter $3$ in
    Krane's text (Ref.~\onlinecite{k88}) and chapter $4$ of Lilley's text (Ref. \onlinecite{l01}).
    %\subsubsection{Nuclear shell model vs liquid drop model(?)}
    %\subsubsection{Binding Energy(?)}
        \subsubsection{Quantum theory of radioactive decays}
        We have learned that the radioactive decay is stochastic in nature and radioactive decay follows
        an exponential law
        \begin{equation}
            \label{eq:lambda}
            \lambda = -\frac{(dN(t)/dt)}{N},
        \end{equation}
        where $N(t)$ is the number of the nuclei at time $t$, and $\lambda$ the decay constant.
        Equation~\ref{eq:lambda} states that the rate of decay from the higher energy state to
        lower energy state is independent of time (since $\lambda$ is constant).
        It corresponds to the fundamental assumption is that the process of radioactive decay is stochastic
        in nature and the probability of decay for any given nucleus is constant,
        regardless of the age of the atom. 
        
        Here we outline an quantum mechanics argument for the time-independent decay probability that is perhaps more
        satisfying.
        We can approximate the nucleus as a particle situated in a potential well with potential given by
        \begin{equation}
            \label{eq:vp}
            \tilde{V}(t) = V + V_p,
        \end{equation}
        where $V$ is the time-independent nucleus potential that gives rise to eigenstates $\ket{\psi_1}$
        and $\ket{\psi_2}$ with $\ket{\psi_2}$ having a lower (expectation value of) energy; $V_p$ is a small
        perturbation that causes the decay.
        Suppose the particle is in a state given by
        \begin{equation}
            \label{eq:state}
            \ket{\psi(t)} = c_1(t)\ket{\psi_1} + c_2(t)\ket{\psi_2},
        \end{equation}
        where $c_1,c_2$ are the coefficients of superposition.
        A decay from $\ket{\psi_1}$ to $\ket{\psi_2}$ is equivalent to
        having $|c_2(t)|^2\gg |c_1(t)|^2$ as time increases. Since we know that time-independent potentials
        only give rise to time-independent states, the fact that Eq.~\ref{eq:state} has time dependence requires
        our perturbation $V_p$ in Eq.~\ref{eq:vp} to be time dependent.
        
        Fermi's Golden Rule \#2 (which we can derive from time-dependent
        perturbation theory)
        gives the transition probability that is to the first order, independent of time,
        \begin{equation}
            \label{eq:trans}
            \lambda = \frac{2\pi}{2\hbar}|\bra{\psi_f}V_p\ket{\psi_i}|^2\rho(E_f),
        \end{equation}
        where $\ket{\psi_i}$ denotes the initial state , $\bra{\psi_f}$ the final state, $V_p$ the perturbation
        potential and $\rho(E_f)$ the density of state of the final state, defined by
        \begin{equation}
            \rho(E_f) = \frac{dn_f}{dE_f}.
        \end{equation}
        In simple terms, $\rho(E_f)dE$ gives the proportion of the particles with energy in the
        really small interval between
        $E_f$ and $(E_f+dE)$.
        Note that as in perturbation theory, $\ket{\psi_i}$ and
        $\ket{\psi_f}$ are both eigenstates of the nuclear potential $V$. As long as
        the decay probability is small (meaning $V_p\ll V$), the result from perturbation theory should hold. Since
        there is no time dependency in Eq.~\ref{eq:trans},the
        decay rate should be independent of time, which is the fundamental assumption made in
        Eq.~\ref{eq:lambda}.
        
        One can easily solve the differential equation given by Eq.~\ref{eq:lambda} to get
        the law of radioactive decay
        \begin{equation}
            N(t) = N_0 e^{-\lambda t},
        \end{equation}
        where $N_0$ gives the number of nuclei in the initial state at $t=0$.


        \subsubsection{Quantifying radioactivity}
        The radioactivity (or simply activity) of a sample $\mathcal{A}(t)$ is given by
        \begin{equation}
            \label{eq:act}
            \mathcal{A}(t)\equiv \lambda N(t),
        \end{equation}
        which is the number of decays occurring in the sample per unit time. The SI
        unit for radioactivity is becquerel (Bq) and it equals one decay per second. Another commonly used
        unit is curie (Ci, with $1\text{ Ci}=3.7\times 10^{10}\text{ Bq}$) and it is defined as
        the activity of radium-266. A commonly found isotope in highly radioactive waste,
        cesium-137 has an activity of $3.215\times 10^{12}$ Bq
        per gram. By contrast, the radioactivity of an adult human body (mainly due to the small amount of 
        potassium-40 and carbon-14), regardless of the weight of the human, is about $8000$ Bq.

        \subsubsection{Different kinds of radiation and shielding}
        Here we review different types of radiation as well as how shielding works. In general, shielding
        of radiation forces the energetic radiated particles (electrons, alpha particles and gamma rays) to
        interact with the shielding material such that the radiation loses much of its energy before it
        reaches a living tissue. It is worth noting that water can serve as a decent shielding due to its
        relatively high density.

        \paragraph{Alpha decay}
        An alpha particle is a helium-4 atom ${}^4_2$He. In an alpha decay, a nucleus emits an alpha
        particle and decays into a different element. The emission of an alpha particle is
        particularly favorable because it does not change the parity of the nucleus and has a particularly
        high binding energy. The process can be represented by
        \begin{equation}
            {}^A_Z X_N \rightarrow {}^{A-4}_{Z-2}X'_{N-2} + \alpha.
        \end{equation}
        Since the alpha particle is charged and massive, it is the easiest radiation to shield.
        alpha particles can be stopped after traveling through 2 millimeters of water. The human skin
        also provides adequate shielding against alpha particles. However, our internal organs are
        vulnerable against alpha particles. Therefore, if the alpha emitter is inhaled or otherwise enters
        one's body, it may cause more harm than other types of radiations because it is more easily stopped
        by the tissues and hence deposits more energy.

        \paragraph{Beta decay}
        There are a variety of beta decay processes. Here we only describe the negative beta decay that is
        relevant in this work. The basic decay process usually comes from the decay of a neutron
        \begin{equation}
            n \rightarrow p + e^- + \bar{\nu}_e,
        \end{equation}
        where $n$ denote  a neutron, $p$ a proton, $e^-$ an electron and $\bar{\nu}$ an electron antineutrino.
        This can also happen inside a nucleus (bound neutron), represented by
        \begin{equation}
            {}^A_Z X_N \rightarrow {}^A_{Z+1} X'_{N+1} + e^- + \bar{\nu}_e.
        \end{equation}
        In a beta decay, the electron emitted carries the most energy. Over 5 centimeter of water usually provides
        good shielding against beta radiation. Metals are extremely effective for shielding beta radiation.
        
        \paragraph{Gamma decay}
        Most nuclear reactions, including the alpha and beta decay, leave the resulting nucleus in an excited state.
        These excited states quickly decay to the ground state via emitting gamma rays with energies in the range
        of $0.1$ to $10$ MeV, which are characteristic of the energy differences between nuclear states.
        This process is very much identical to its atomic counterpart. As high energy massless particles, gamma
        radiation is the hardest to shield, and the thickness of the shield depends on the energy of the photons.
        Usually it takes 5 meters of water, or about 2 meters of concrete,  to reduce the energy of the photon
        to one billionth of its original energy. Metal is usually more effective against gamma radiation since
        gamma radiation is a form of electromagnetic wave.\cite[chap. 1]{l01}

    \subsection{Biological Effects of Radiation}
        The biological effects of radiation have been an important topic for decades. A large
        number of studies took place after the two nuclear bomb explosions in Japan in 1945
        and after disastrous nuclear plant accidents like Chernobyl. Therefore, we have a relatively
        thorough understanding of the consequences of mass exposure to radiation. However, relatively
        little is known about the long-term effects of exposure to small doses, including subtle biological
        changes that lead to cancer and genetic defects. Here we briefly summarize the known effects.
        \subsubsection{Quantifying radiation on living tissues}
        The biological impact of radiation depends on the type, power and energy of the radiation, the
        exposure time, and the part of the body exposed. Various quantities have been introduced to
        measure the amount of radiation and the biological impact of radiation. Here we review three
        of these quantities: the absorbed dose, the equivalent dose and the effective dose.
        The SI units for all of these quantities are equivalent to J/Kg, the amount of energy deposited into
        per unit mass of living tissue. However, this unit is called gray (Gy) for absorbed dose but sievert
        for equivalent dose and effective dose.

        The amount of radiation is usually characterized by the amount of energy it carries. Since one of
        the earliest understood effects of radiation is its ability to ionize gas, one can quantify the
        strength of the radiation with the ionizing energy of the gas.
        Absorbed dose~($D$) quantifies the 
        radiation energy absorbed per unit mass. The modern unit is Gray(Gy)
        and $1\text{Gy}=1\text{J kg}^{-1}$.\cite{u16}

        Biological effects not only depend on the total dose applied but also the rate at which the 
        dose is applied, because mechanisms exist within organisms that allow molecules like deoxyribonucleic acid
        to be repaired if they are not excessively damaged. Therefore {\it dose rate} is also a useful
        measure of radiation when considering its biological effect.

        For an external $\gamma$-ray source emitting $\mathcal{A}$ million photons per second 
        of energy $E_{\gamma}$
        (in MeV), one can approximate the dose rate by 
        \begin{equation}
            \frac{dD}{dt}(\mu\text{Sv h}^{-1}) \approx \frac{\mathcal{A}\text{(MBq)}\times E_{\gamma}\text{(MeV)}}
            {6\times [r\text{(m)}]^2}.
        \end{equation}
        The approximate linear dependence on $E_{\gamma}$ is valid from less than $0.1$ MeV to several
        MeV and therefore should be good for all sources but those generated in an accelerator.\cite{my68}
        The inverse-squared dependence on distance is hardly surprising.

        However, absorbed dose does not take into account different biological implications of different types
        of radiation.
        For example, the same dose of alpha particle can do more damage to body than gamma rays. The equivalent
        dose ($H$) is defined as
        \begin{equation}
        H=w_R \times D_R,
        \end{equation}
        where $D_R$ is the absorbed dose and $w_R$ the weighting factor of a given kind of radiation $R$
        (see Table~\ref{tab:eq} for a list of weighing factors). If there are multiple types of radiation 
        present, the equivalent does is given by the weighted sum over all contributions. The SI unit for
        equivalent dose is called the sievert (Sv) and $1\text{ Sv}=1\text{ J kg}^{-1}$.
        \begin{table}
            \centering
            \caption{Weighting factors for different types of radiation\cite{icrp74}}
            \begin{ruledtabular}
                \begin{tabular}{l c c}
                Type of radiation & Energy range & Weighting factor, $w_R$\\
                \hline
                Photons, electrons & All energies & 1\\
                Neutrons & $<10$ keV & 5 \\
                         & $10-100$ keV & 10 \\
                         & $100$ keV-$2$ MeV & 20 \\
                         & $2-20$ MeV & 10 \\
                Protons & $<20$ MeV & 5 \\
                Alpha particles, fission fragments, heavy nuclei & & $20$\\
            \end{tabular}
            \label{tab:eq}
            \end{ruledtabular}
        \end{table}

        Finally, certain organs and parts of the body are more vulnerable to radiation than the rest.
        Therefore, effective dose~($E$) is defined as the equivalent dose weighted by the body part
        radiated
        \begin{equation}
            E = w_T H_T,
        \end{equation}
        where $w_T$ is the body tissue weighting factor, and $H_T$ the equivalent dose on the tissue
        at question.
        Evidently, if the radiation affects multiple tissues, the effective dose will be the weighted
        sum of the equivalent dose over individual body parts. Some of the recommended values of tissue
        weighting factors are given in Table~\ref{tab:eff}. A weighing factor of $1$ is equivalent to the
        whole body absorbing the radiation uniformly. The SI unit of effective dose is sievert (Sv),
        identical to that of equivalent dose.
        \begin{table}
            \centering
            \caption{Weighting factors for individual organs\cite{icrp74}}
            \begin{ruledtabular}
                \begin{tabular}{l c}
                Tissue & Weighting factor, $w_T$\\
                \hline
                Gonads & $0.20$\\
                Red bone marrow & $0.12$ \\
                Liver & $0.05$ \\
                Skin & $0.01$ \\
                \end{tabular}
                \label{tab:eff}
            \end{ruledtabular}
        \end{table}
        \subsubsection{Effects of radiation on human}
        Short-term effects of radiation are usually caused by extensive cell death or cell damage.\cite{u16}
        These effects are deterministic, meaning that below a certain threshold the effect does not occur and
        the severity of the effect depends on the dose. Some examples are gastrointestinal damage and impairment
        of fertility.\cite{u16, l01}

        Delayed health effects occur a long time after the exposure and are usually stochastic. There is
        no apparent threshold and the severity is usually independent of the dose but
        the chance of occurrence depends on the dose.\cite{u16,l01}
        For example, according to a review done by the UN Scientific Committee on the Effects of Atomic Radiation
        (UNSCEAR), the additional chance of dying of cancer due to radiation exposure above $100$ mSv is
        about 3 to 5 percent per sievert.\cite{unscear16}

        Hereditary effects of radiation can occur if the reproduction cells were damaged by radiation.
        UNSCEAR estimates that risk for severe hereditary effect is about $0.3-0.5$ percent per gray to
        the first generation following the radiation exposure.\cite{u16, unscear12}

%        \subsubsection{Limitations of current studies}
%        \red{TODO}

    \subsection{Fission}
    \label{sec:fission}
        Nuclear Fission is the process in which a heavy nucleus splits into smaller fragments and releases
        a huge amount of energy. It is the main physical process taking place in a nuclear reactor. It is
        also the main mechanism that generates the most radiative wastes that we will discuss later
        in this work. Therefore, we should first familiarize ourselves with fission.
    \subsubsection{Neutron absorption}
    \label{sec:capture}
    Thermal (slow) neutrons induce fissions in a reactor.
    Since a neutron does not carry any charge, it experiences a much smaller potential when it interacts
    with a nucleus. In the nuclear absorption reaction, a neutron crosses the potential barrier to form
    an excited nucleus.
    The nuclear absorption reaction is the most crucial type of reactions that takes place in
    a nuclear reactor. As neutrons travels through materials, they react readily with nuclei.
    For fast neutrons, many reactions are possible. But for slow or thermal neutrons (as in the
    case of a nuclear reactor), neutron absorption is the primary mode of interaction.
    Depending on the path that the compound nucleus takes to return to its ground
    state, neutron absorption is classified into radiative capture and neutron-induced
    fission(see section~\ref{sec:fission}).
    
    In a radiative capture process, the compound nucleus emits photons in the form of gamma rays, whereas in neutron-induced
    fission, the compound nucleus splits into fragments.
    In fact, not all nuclei can undergo fission.
    Therefore, radiative capture is the only possible neutron absorption process for
    nuclei that are not fissionable.\cite{lb01} We will explore neutron-induced fission further in
    section~\ref{sec:fission}. Here we give an example of radiative capture. When a stable form
    of gold ${}^{197}$Au absorbs a neutron, it forms a highly unstable isotope ${}^{198}$Au in its
    excited state and quickly decays into its ground state by emitting $\gamma$ rays. This can be
    written as
    \begin{equation}
        {}^{197}\text{Au} + n \rightarrow {}^{198}\text{Au} + \gamma.
    \end{equation}
    Moreover, the isotope ${}^{198}$Au can spontaneously decay into ${}^{198}$Hg via beta decay. This is not
    uncommon, as the nucleus formed by neutron capture is usually unstable.
        \subsubsection{Fission and fission products}
        Conjectured by Bohr and proven by Alfred O. Nier in 1939,
        the fission of uranium-235 induced by a thermal neutron
        is the best-known example of induced fission. 
        The neutron capture produces a compound nucleus ${}^{236}U$ that is in an excited state.
        Sometimes the nucleus can decay by $\gamma$ emission in capture reaction.
        But most of the time, the excitation energy deforms the nucleus (see Fig.~\ref{fig:fission}).
        When the
        deformation reaches a certain point, the Coulomb force overcomes the short-ranged nuclear
        force (residual strong force) and the nucleus disintegrate into two large fragments and several neutrons. The
        positively-charged fragments(fission products) attain approximately $170$ MeV of kinetic energy as
        Coulomb force continues to drive them apart.\cite{l01}
        A typical reaction is described by
        \begin{equation}
        {}^{235}_{92}\text{U} + n \rightarrow {}^{92}_{36}\text{Kr} + {}^{142}_{56}\text{Ba} + 2 n.
        \end{equation}
        The energy liberated is about $180$ MeV. The combination of fission products is not
        unique. \cite{w98, gc01}
        Very often, photons, electrons and neutrinos are also emitted through beta and gamma decay.
        On average, $2.5$ neutrons are produced per fission.\cite{l01} 
        \begin{figure}[h]
            \centering
            \includegraphics[width=0.8\textwidth]{fission.png}
            \caption{Illustration of the fission process for uranium-235.\cite{l01}}
            \label{fig:fission}
        \end{figure}

        \subsubsection{Fission energy budget}
        We can estimate the amount of energy released in fission by considering the binding energy difference
        and the properties of the product. Essentially, any energy carried by particles that can be slowed down 
        in the reactor can be harvested as heat to generate power.

        The binding energy per nucleon is about $7.6$ MeV u${}^{-1}$ for uranium and about $8.5$ MeV u${}^{-1}$
        for a nucleus with atomic number around $117$. Hence, the change of binding energy per nucleon is
        $0.9$ MeV u${}^{-1}$, which amount to $212$ MeV for uranium-235.
        About $87\%$ of the total energy is promptly emitted because
        the fission products, carrying over $90\%$ of the total energy, travel for only a fraction of a
        millimeter before they are stopped. Besides, fission neutrons emit about $2$ MeV of energy by
        successive scattering in the reactor.
        About $13\%$ of the total energy goes into radioactivity. Energy carried by electrons and photons is
        convertible to heat but the neutrinos, carrying about $50\%$ of the radiation energy, are not stoppable.

        Henceforth, it is generally accepted that about $220$ MeV per fission is recoverable for energy conversion.
        The energy generated by the fission of uranium-235 is 
        about 2.5 million times the energy produced by burning coal of the same mass.\cite{e17}
%        \subsubsection{Chain reaction??}

\section{Nuclear waste: classification and properties}
\label{sec:waste}
    Different organizations have slightly different classification schemes for nuclear waste.
    However, their definitions of high-level waste (HLW), the most detrimental kind, are relatively consistent.
    The International Atomic Energy Agency (IAEA) classifies nuclear waste by considering the activity
    content and the half-life of the radionuclides.
    Activity content is a measure of how much radioactivity $\mathcal{A}(t)$, as defined in Eq.~\ref{eq:act},
    a given amount of waste contains.
    A form of waste may possesses a high activity content due to
    a high concentration of radionuclides or the presence of radionuclides with high activity.\cite{iaea09}
    Each form of waste, depending on the half-life, activity content, and temperature, requires different
    disposal methods.

    We will focus on high-level waste (HLW, defined below) in the rest of this work.
    In this section, we will first examine the composition and properties of HLW and then
    comment on other categories of waste and disposal methods (following the NRC classification scheme). 
    \subsection{High-level waste(HLW)}
    High-level Wastes (HLW) has high activity contents and usually generates a large amount of
    heat as it radiates. The typical source of HLW is spent fuel from the nuclear reactor.
    In fact, since there is no commercial reprocessing facilities in the US (see section~\ref{sec:reproc}),
    almost all of the HLW is in the form of spent nuclear fuel from the reactor. In a nuclear reactor,
    fission of uranium-235 produces lighter radionuclides like cesium-137 and strontium-90, which
    account for most of the heat and penetrating radiation (gamma radiations). In addition to
    the fission process, some uranium-235 atoms go through radiative capture
    (see section~\ref{sec:capture}) and produce radionuclides that are
    heavier than uranium (``transuranic'') such as plutonium. The transuranic atoms do not generate
    as much heat or radiation from their subsequent decay as the fission products, but they tend to have
    a much longer half-life.
    For example, strontium-90 and cesium-137 have half-lives of about $30$ years, whereas
    the transuranic plutonium-239 has a half-life of $24,000$ years.

    \subsection{Transuranic waste (TRU)}
    \label{sec:tru}
    Transuranic waste is defined as materials that do not belong to high-level waste and are
    contaminated by alpha-emitting radionuclides with
    sufficiently long half-life ($>20$ years) of elements with atomic number $92$ (atomic number of uranium)
    or larger and concentration of activity larger than $3700$ Bq/g.\cite{j83,s01} The activity
    of TRU is usually low but it remains active for long period of time.

    \subsection{Low-level and intermediate-level waste (LLW and ILW)}
    Low-level waste (LLW) has relatively little radioactivity, produces almost no heat,
    and contains practically no
    transuranic elements. Examples are papers and tools used in cleanup operations in the reactor.
    Most LLW does not require shielding and may be buried in landfill 
    facilities after on-site short-term decay storage.
    Some LLW (for example, metal nuclear fuel cladding) may have a slightly
    higher radioactivity such that they are classified
    as intermediate-level waste (ILW). ILL are solidified in concrete and buried in shallow
    repositories.\cite{s01} However, US regulations do not make the distinction
    between LLW and ILW but different disposal methods are practiced.\cite{nrc09, s01}

    \subsection{Uranium tailings}
    Uranium mill tailings from uranium mills are another form of waste with low activity, but
    they are not usually classified as LLW. The tailing is what is left after the uranium has been
    extracted from the ore. The decay of uranium-238 from the mill decays into
    radioactive thorium and radium and these radionuclides stay in the tailing. Notably, radium
    can decay into radon gas which is an alpha-emitter. Hence, the alpha radiation can do additional
    damage to body if the radon enters the body as one breathes.
    The uranium tailings leave the uranium mills in the form of a liquid sludge
    and can dry. Therefore, it is necessary to prevent the uranium tailings
    from contaminating water and air.
    
    \section{Spent fuel management}
    \label{sec:temp}
    Spent fuel from nuclear reactors constitutes the major source of HLW. Spent fuel pools at the
    reactors hold spent fuel cores to cool them down and isolate them for a few years. After that,
    reprocessing can occur to extract the most radioactive plutonium and uranium isotopes and
    convert the waste into more manageable form (vitrification). However, reprocessing of commercial
    nuclear spent fuel did not
    happen in this country for various historical reasons. 
    \subsection{Spent fuel pools}
    The heat output of the spent fuel decreases by $99\%$ after the first year and by another factor of $5$
    by the 5-year mark.\cite{aa12} Therefore, short-term cooling of spent fuel is crucial before
    further disposal or reprocessing can be done.
    US nuclear power plants generally have an operation cycle of about 2 years. At the end of each operation cycle,
    one-quarter to one-third of its fuel assemblies (spent fuel) are removed.
    Most of the nuclear industry's spent fuel is now stored in spent fuel water pools ``at-reactor (AR)'',
    which means either within the reactor building or in an adjacent spent fuel building linked to the reactor
    AR pools minimize the difficulty of transporting spent fuel.\cite{iaea99}
    US has generated about $65,000$ metric tons of spent fuel, of which $75\%$ is stored in AR pools.\cite{a11}
    by a transfer tunnel. 
    Spent fuel pools  are usually about 15-meter deep borated water pools, with storage rack at the bottom to hold the fuel assemblies.
    The borated water contains boron which are more prone to neutron captures.
    The water absorbs the heat and shields the radiation. Water is the chosen medium because it is cheap and
    can provide cooling by circulation. Moreover, water can serve as a shielding from radiation, because it has a
    fairly high density and hence forces the radiated particles (especially the neutrons) to interact with
    the water molecules and lose energy. Water temperature is maintained
    below 50 $^\circ$C via circulation and cooling. The integrity of the storage can persist well beyond 
    50 years.\cite{a11, iaea99}

    \subsection{Reprocessing: an option}
    \label{sec:reproc}
    After spending some time in the nuclear fuel pools and cooling down, the spent fuel may be ready
    for reprocessing.
    Spent nuclear fuel contains a large amount of plutonium and uranium that are suitable as nuclear fuel
    or fissile weapon material (plutonium-239). Therefore, reprocessing the spent fuel generates plutonium and
    uranium that may be recycled. The US currently does not have any commercial nuclear fuel
    reprocessing facility due to the demonstrated high cost of reprocessing
    during the West Valley Demonstration Project and water contamination issues at the Hanford Site.
    Concerns about proliferation issues also arise, as the 
    reprocessed plutonium-239 is the primary fissile element in the production of nuclear weapons.\cite{aa12}
    On the other hand, France (the world leader in reprocessing technology), India, Japan, Russia, the UK
    and China (since 2010) continues to reprocess spent fuel. To understand the details of reprocessing
    requires advanced chemistry and is beyond the scope of this work. However, the issue remains interesting
    and hence we provide an overview of the process as well as some of the related concerns.

    Almost all of the reprocessing plants use the PUREX (Plutonium Uranium Redox EXtraction) process
    to reprocess spent fuel. This process utilizes the differences in oxidation potential 
    between the heavy elements (uranium and plutonium) from the other lighter fission products.
    The process involves three steps: (a) adding appropriate solvent (usually nitric acid) to absorb and oxidize 
    plutonium and uranium while leaving out other elements;
    (b) adding a reducing (de-oxidizing) agent that reduces (de-oxidizes) only the
    oxidation state of plutonium to one that is insoluble
    and thus separable; and finally (c) extracting the plutonium from the mixture. In this way
    the oxidized plutonium and uranium are separated and can be reduced back to reusable radionuclides.\cite{lb01} 
    The extracted plutonium then returns to the reactor and transforms into short-lived fission products
    by fission. Un-reprocessed fuel usually takes 10 times as much time to reach the safe
    activity level ($\sim 10^5$ years) as reprocessed fuel ($\sim 10^4$ years). Moreover, since fission products
    make up a small portion of the spent fuel, reprocessing greatly reduces the mass of the waste. The reprocessed
    waste can be turned into liquid form, mixed with frit (the substance from which glass is made) and then
    vitrified (made into glass). The vitrification process immobilizes the radioactive particles and makes
    the waste much more safe and manageable, but vitrification can only be applied to reprocessed spent fuel.

    It is worthwhile to note that the PUREX processed was designed specifically to separate plutonium for
    bombs. Therefore, literally any other reprocessing approach would be less prone to proliferation.
    Researchers are actively investigating alternatives to the PUREX technology. For example, the UREX+ (for
    uranium extraction) technique extracts the uranium but leaves the plutonium mixed with other
    radioactive materials. Another popular alternative, pyrochemical reprocessing, inserts giant electrodes
    into the chemical bath of nuclear waste and electroplates the plutonium and other transuranic elements onto
    the electrode. However, none of these alternatives have been put into practice due to the high construction
    and operating costs of reprocessing facilities.\cite{aa12}

    %\red{fast neutron reactors}

    \subsection{Dry cask storage}
    Spent fuel pools have very limited capacities. 
    Overground dry cask storage provides a low-cost temporary storage before the spent fuel is disposed of.
    After cooling in the spent fuel pools for at least a year, the spent fuel can be stored in dry casks filled with inert gas. 
The dry casks are often metal cylinders welded to prevent leakage. Each cask is surrounded by
    additional steel and concrete to provide shielding.
    The heat from the cask is removed by passive
    cooling (air circulation). The NRC is hoping that the casks will
    last safely for as long as 100 years, yet it has been observed that corrosions and cracking occur in 30 years
    or less.\cite{aa12} As such, without the prospect of a permanent geological repository,
    overground dry cask storage does not seem to be a long-term solution. The doubts were expressed
    during the Nuclear Waste Technical Review Board 2009 meeting \cite{nwtb09, aa12}
    \begin{displayquote}
    Leaving the spent fuel on-site for extended periods of time was never intended and is not
    responsible. (Dry cask manufacturer)

    It is not ethical, basically to plan for long-term storage without pursuing a well defined repository
    program. (Department of Energy employee speaking as a citizen)
    \end{displayquote}
    Even more striking is the fact that more than $75\%$ of the spent fuel are still stored in densely
    packed pools
    that are close to or over their limiting capacity.\cite{a11,aa12}
    Therefore, there is clearly a need for a long-term disposal plan, which we will explore in
    the next section.

\section{Disposal of HLW}
\label{sec:disposal}
    The time frame of nuclear waste disposal is about $10^4-10^6$ years, after which the radioactivity
    of the waste is predicted to decline to a safe level.
    Deep geological repositories that are stable over geological timescales naturally offer a good option for disposal.
    In fact, it is accepted by the international science community that geological repositories in a stable environment over 300 meters below ground
    is safer than indefinite overground storage of HLW, reprocessed or not.\cite{fmr11}
    In this section, we first use the Yucca Mountain Nuclear Waste
    Repository as a case study to understand the requirements for a permanent HLW disposal site and the debate
    surrounding it.
    Then we look at an alternative to a permanent deep geological solution:
    deep boreholes storage.
    \subsection{Deep geological repository: Yucca Mountain}
        The defining characteristic for a nuclear waste disposal site is its ability to minimize human
        exposure to radiation over the next ten thousand to a million years or more.
        For a deep geological repository, this means preventing the release
        of radionuclides to the ground water or the surface.
        Due to the amount of radiation the repository would contain, it would be difficult to perform
        maintenance once the waste was deposited. Therefore, scientists and engineers are challenged
        to predict the geological environment of the Yucca Mountain over a million years and come up
        with a design that they will have very little chance to alter once it is implemented.

        The Yucca Mountain Nuclear Waste Repository was designated by the Nuclear Waste Policy Act
        of 1987 to be the nation's permanent deep underground repository for spent nuclear fuel. It is
        located near the Nevada Test Site in Nevada. The project has been highly contested since its
        inception and the Obama administration ended federal funding for the project in 2011 for political,
        not safety or technical, reasons. We will take a glimpse at the suitability assessment of the site
        as a nuclear waste repository to get a sense of its complexity.
        \subsubsection{Geological stability of Yucca Mountain}
        The chief concern of the suitability of Yucca Mountain was about water. If moist air comes to
        contact with the waste, the waste may get oxidized and become chemically and physically unstable, thereby
        releasing radionuclides.
        The community seemed to agree that the cladding will have to carry the primary
        responsibility of preventing the oxidation of the waste, but predicting the performance of some
        engineered product over the timescale of a million years is not something with which we have had much 
        experience. Over time, the radiation from the waste will also alter the rock (by altering its composition
        and creating fractures) and the water in the rock.\cite{m06}

        Many other factors are considered. Climate change will influence precipitation and thus the
        rate at which water percolates the rock and reaches the waste. 
        Tectonic motion may change how much the rocks are fractured and thus the percolation rate.
        Volcanic activity, meteorite hit and earthquakes may seriously undermine the integrity of the repository.
        Hot upwelling water from below the repository may corrode the cladding and release radionuclides into
        the ground water system. It is impossible to address all of these concerns in this work, but we will look
        at how the risk assessment is carried out in the case of volcanism.
        \subsubsection{Total System Performance Assessment (TSPA)}
        The Department of Energy (DOE) created Total System Performance Assessment (TSPA) as a huge probabilistic
        computer model that predicts the potential annual dose to the potential receptor.
        Essentially, the TSPA takes a huge set of parameters characterizing features that DOE deems as important,
        carry out the geophysical and geochemical simulations,
        and outputs the expected annual dose leaked from the repository.
        DOE expects that
        uncertainty is fully integrated into the analysis done by TSPA such that even if the model itself change
        (for example if new parameters and factors are included), the results from TSPA will not change
        significantly.e\cite{cv14,ocrwm02} 

        \subsubsection{Risk and uncertainty estimation: volcanism}
        \label{sec:volcano}
        Yucca Mountain is composed of volcanic rocks created from a series of large-volume eruptions between
        fifteen million and eight million years ago. However, it is the small-volume eruptions, the last of
        which occurred about $75,000$ years ago, that would pose risk to the repository in its life span.
        The central question is whether the risk of volcanic eruption is acceptable. Mathematically,
        according to the Environmental Protection Agency (EPA)'s 1993 guideline,
        if the probability of a volcanic eruption seriously damaging the repository is less than
        one in one hundred million per year, it is acceptable.\cite{epa93}
        Of course, one has to also consider consequences of a potential radiological releases if an eruption
        occur. Here we focus on the estimation of occurrence probability.

        The probability that a future volcanic eruption disrupting a repository is given by
        \begin{equation}
            \label{eq:prob}
            P(D) = P(E2|E1)P(E1),
        \end{equation}
        with the standard probability notation: 
        $P(A)$ means probability of event $A$ happening;
        $P(A|B)$ gives the probability of event $A$ happening given that event $B$ happened;
        $D$ denote the disruption event,
        $E2$ the event that the eruption affects the repository, $E1$ the event that an eruption occurs.
        Essentially, we are looking at the probability of a volcano occurring and affecting the repository.

        \begin{figure}[h]
            \centering
            \includegraphics[width=0.9\textwidth]{volcano.png}
            \caption{A flowchart for the volcanism risk estimation problem. The oval-shaped
            parts are all probabilistic. \ref{sec:volcano} \cite{cv14}}
            \label{fig:volcanos}
        \end{figure}

        The information necessary to estimate the probability is listed in the flowchart Fig.~\ref{fig:volcanos}.
        The boxes represent decision parameters that depends on the judgement of the decision makers and
        the scientists. For example the ``Volcanic Zone Models'' should be picked by the scientists, and
        ``Repository Area'' decided by policy makers.
        
        The ovals are probabilistic parameters in which uncertainty
        is represented as a probability distribution over a range of possible values.
        Each oval is a complex computational model.
        The large bolded ovals are the $E1$ and $E2$ parameters in Eq.\ref{eq:prob}.
        The $E2$ oval makes
        use of the events simulated in $E1$ to calculate the conditional probability $P(E2|E1)$ and output
        the disruption probability $P(D)$ to TSPA, which calculates the expected leaked dose over the
        life span of the repository. Here we look in some detail at the calculation of those parameters.
        \paragraph{Monte-Carlo simulation}
        Estimation of uncertainty is key in this problem, because over the timescale of a million years,
        a small uncertainty would amount to large measurable change. Recall the traditional method of
        propagation of uncertainty. Let $\delta_X$ denote the uncertainty in variable $X$. For a function
        with analytic form $y(x_1,x_2,...x_n)$, the uncertainty in $y$ is given by
        \begin{equation}
            \delta_y = \sqrt{\sum\limits_{i=1}^n \left(\left.\frac{\partial y}{\partial x_i}
            \right\rvert_{x_i} \delta_x\right)^2},
        \end{equation}
        which uses a linear approximation of $y(x_1,x_2,...,x_n)$ near the measured values of
        $(x_1,x_2,...,x_n)$.
        However, if the uncertainty of the input parameter if expressed as a distribution, or if $y$ is
        not an analytic function, then the propagation of uncertainty method fails. In our volcano disruption
        problem, $y$ can be seen as the complex computational model like TSPA. However, there is a
        straightforward algorithm called Monte-Carlo simulation that can estimate the uncertainty of $y$.
        The steps of the algorithm is as follows.
        \begin{itemize}
            \item Generate a large number of $(x_1,x_2,...x_n)$ samples according to their respective distributions.
            \item For each $(x_1,x_2,...,x_n)$ sample, calculate a $y$ value (with the computational model).
            \item Aggregate all the $y$'s generated to estimate its distribution or uncertainty.
        \end{itemize}
        This method usually converges reasonably fast -- the error goes down as $1/\sqrt{N}$ where $N$ is the number
        of samples generated. It is the algorithm of choice in Fig.~\ref{fig:volcanos} to
        propagate the uncertainty information in one oval to another.
        \paragraph{Recurrence rate} The recurrence rate $E1$ only uses information from the ``Event Count''
        oval, which estimates the true number of eruptions in the Yucca Mountain region in a given time range
        from observed data. The choice of time range is not trivial: longer time range certainly gives us
        more data and hence improved statistics, but longer time interval dated far back may fail to represent
        future event rates since the geology evolves. The best approach seems to be choosing time interval
        based on geological cycles.

        Finally, we need the undetected volcanic intrusion rate to establish the recurrence rate. Volcanic
        intrusions are essentially underground eruptions that can go undetected. The DOE field studies conclude
        that intrusions are possible but are presumed to be associated with surface eruption in space and time. 
        By contrast, the NRC argues that intrusions need not be associated with eruption and there
        may be three to five times as many undetected intrusions as detected surface eruptions. The divergence
        of the two agencies' viewpoints partly attributes to their different estimated recurrence rate.
        From different studies adopting different input models, estimated recurrence rate ranges from
        $1.5\times 10^{-6}$ to $8\times 10^{-6}$ events per year. These estimates translate to an estimated
        time between events from 125 thousand years to 250 thousand years. We can see that the choices of
        models introduce a significant uncertainty to the estimation.\cite{cv14}

        \paragraph{The probability of repository intersection}
        Given a probability distribution for $E1$, we can estimate $P(E2|E1)$ -- the probability of
        a given event intersecting the Yucca Mountain site. It is simply the ratio of the area of
        the repository to the entire volcanic zone of a given event. Since the area of the repository
        has not been decided, a probability distribution over a range of areas is used.
        However, since not enough volcano events have been observed in the region, the model
        to predict the size of the volcanic zone is not well determined.
        Different models predict $P(E2|E1)$ ranging from $0.015$ to $0.0008$, which again has a high
        uncertainty.

        \paragraph{The probability of a disruption} The probability distribution of $E1$ and $E2$
        are combined through a Monte-Carlo simulation to produce the disruption probability/rate $P(D)$
        at the Yucca Mountain repository. DOE estimates a mean disruption rate of $1.5\times 10^8$ with
        a 90 percent confidence interval of $5.4\times 10^{-9}-4.9\times 10^{-8}$. Hence the DOE's result
        states that most likely the disruption won't happen more than once every 20 million years.

        The NRC emphasizes the importance of the bounding nature of the problem and uses only a subset
        of the DOE model that reflect conservatism. It concludes a disruption rate of
        $10^{-7}$ to $10^{-8}$ per year. Furthermore, it argues that it should not distinguish the likelihood
        of disruption rate of different values within this range. Therefore it picks the upper bound
        of $10^{-7}$ as the estimated event rate, representing the worst case estimate.

        Both the NRC and the DOE conclude with a very low volcanic disruption rate that is below the EPA
        threshold. Nevertheless, the consequences of such a disruption event is still not well understood.
        If the leaked dose is low in such an event, then the risk is negligible. If the leaked dose is high,
        then the NRC estimate of $10^{-7}$ events per year would necessitate further investigation into such
        a scenario.\cite{cv14}
        \subsection{Deep borehole disposal}
        Deep borehole disposal utilizes deep and narrow boreholes to dispose of HLW. The boreholes
        will place the cladded waste as much as 5 kilometers under the surface of the ground and seal it
        with concrete clay.
        The engineered clad, along with the granite rock with few cracks on which the borehole is drilled,
        provides
        good isolation for the radionuclides in the waste (See Fig.~\ref{fig:borehole}).
        \begin{figure}[h]
            \centering
            \includegraphics[width=0.8\textwidth]{borehole.jpg}
            \caption{Illustration of borehole disposal, with deep repositories' depths for comparison.\cite{c15}}
            \label{fig:borehole}
        \end{figure}

        Deep boreholes have several advantages over deep geological repository. For example, suitable
        rocks can be more readily found for boreholes (since it does not take up a lot of space and can be
        of smaller scale),
        whereas large repository sites are hard to identified. \cite{b09,c15}
        However, the borehole solution has its own limitations.
        Drilling a hole as deep as 5 kilometers poses a daunting engineering challenge.
        Moreover, reactor rods from commercial nuclear plants in the U.S. usually size up to 2 meters in diameter,
        which would be too large for the borehole and require extensive repackaging.
        Nevertheless, deep borehole disposal remains a valid option.
        Sandia lab has embarked on a five-year plan to experiment with the deep borehole design.\cite{c15}

        %\subsection{reversible disposal}

\section{Discussions}
Nuclear waste management and disposal certainly is becoming a more and more pressing issue.
However, given the technical and political complexity involved in assessing the suitability
of a permanent repository site,\cite{m14} it seems like dry cask storage should be promoted extensively
before a permanent repository is ready. Large-scale dry cask storage would ease the pressure on the
spent fuel pool while easing the pressure on DOE and NRC to rush the scientific analyses.
But still, we have yet to know the durability of dry casks.

In terms of scientific analyses for a geological repository, much has yet to be done.
On the performance analysis front, we need
to think about how to model without enough data and how to deal with orders of magnitude uncertainty.
Perhaps some other analysis methods are required. Building and comparing the performance of 
small retrievable repositories that store
nuclear waste for about 100 years may give us insights on the stability of different geological sites.

Newer reprocessing methods (as mentioned in Sec.~\ref{sec:reproc}) are being proposed with
fast neutron reactors as a solution to the proliferation issue with current reprocessing method.
A fast neutron reactor uses neutrons of higher energy to induce the fission of plutonium-239
such that the spent nuclear fuel would contain very little plutonium-239. Reprocessing and
vitrification techniques can significantly reduce the volume of the waste and make it more
manageable. With reprocessing, deep boreholes storage may become a more feasible disposal solution 
for commercial reactor waste.

%\begin{acknowledgments}
%Thanks Joel and peeps. Arjendu and M. V. Ramana.
%\end{acknowledgments}
\newpage
\bibliographystyle{aip}
\bibliography{refs}
%\newpage
%\appendix
%\section{Derivation of Fermi's Golden Rule \#2}
%\label{a:fermi}
\end{document}
